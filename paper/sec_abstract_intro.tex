% !TEX root = atlas_iros_16.tex


%%%%%%%%%%%%%%%%%%%%%%%%%%%%%%%%%%%%%%%%%%%%%%%%%%%%%%%%%%%%%%%%%%%%%%%%%%%%%%%%
\begin{abstract}
% motivation
Soft robotics methods such as im\-pe\-dance control and reflexive collision handling have proven to be a valuable tool to robots acting in partially unknown and potentially unstructured environments. 
Mainly, the schemes were developed with focus on classical electromechanically driven, torque controlled robots. 
There, joint friction, mostly coming from high gearing, is typically decoupled from link-side control via suitable rigid or elastic joint torque feedback.
% problem
Extending and applying these algorithms to stiff hydraulically actuated robots poses problems regarding the strong influence of friction on joint torque estimation from pressure sensing, i.e. link-side friction is typically significantly higher than in electromechanical soft robots.
% contribution
In order to improve the performance of such systems, we apply state-of-the-art fault detection and estimation methods together with  observer-based disturbance compensation control to the humanoid robot Atlas. 
With this it is possible to achieve higher tracking accuracy despite facing significant modeling errors. 
Compliant end-effector behavior can also be ensured by including an additional force/torque sensor into the generalized momentum-based disturbance observer algorithm from \cite{DeLucaAlbHadHir2006}.
\end{abstract}
%
%\begin{keywords}
%generalized momentum based disturbance observer, collision detection, Atlas robot, joint impedance controller, observer feedforward, DARPA Robotics Challenge
%\end{keywords}


%%%%%%%%%%%%%%%%%%%%%%%%%%%%%%%%%%%%%%%%%%%%%%%%%%%%%%%%%%%%%%%%%%%%%%%%%%%%%%%%
\section{\large Introduction and State of the Art}
\label{sec:intro}
%%-- "General context"
% TODO: Zitate für diesen Absatz benötigt?
% Compliant manipulation und Reflexe als Mittel zur Erreichung eines sicheren Verhaltens
Compliant manipulation and appropriate reflex reactions to collisions have been an active research field over the last decades, opening the door to safer and more autonomous robot applications \cite{Haddadin2014,BicchiTon2004,DeLucaFla2012a}.
Human-friendly robot behavior has to be ensured not only for industrial robotic co-workers, which are typically serial chain manipulators, but also in future healthcare, rescue or even personal robotics applications, where mobility is essential. 
At the same time, compliance control and reflexive contact handling are sought to be essential features in damage protection for systems as the Boston Dynamics Atlas hydraulic robot \cite{2014:JFR-ViGIR-DRC-Trials,ConnerKohRomStu2015}.

% Möglichkeiten zur Umsetzung der Nachgiebigkeit: Inhärent oder aktiv geregelt.
Essentially, softness is either achieved by an inherently compliant structure \cite{PrattWil1995,BicchiTon2002} and/or active compliance control via high-fidelity joint torque feedback. 
% Impedanzregelung als eine Möglichkeit, nachgiebiges Verhalten zu erreichen
One of the most prominent control concepts to implement compliance is impedance control.
It was introduced in \cite{Hogan1985} and extended to flexible joint robots e.g. in \cite{Albu-SchaefferOttHir2007,Ott2008}. 
% Bezug Elektromechanische Roboter
However, up to now the schemes were mainly applied to electromechanically driven robots.
% Bezug Hydraulische Roboter und andere Regelungsmethoden als Impedanzregelung (Beispiele nur für Ganzkörper- oder Ganzarmkonzepte. Keine einzelnen Gelenke, das ist mehr Fokus des Humanoids-Beitrages)
For hydraulic humanoid robots, basic compliance control schemes were implemented on the SARCOS humanoid, focussing on balancing and contact Jacobians \cite{Hyon2007SARCOScompliance}, on constraint handling via quadratic programming \cite{HerzogRigGriPas2014} or with LQR feedback gain optimization \cite{MasonRigSch2014}. 
The concept of admittance control (position-based impedance control) has been implemented on the Atlas robot in \cite{Lee2014atlasadmittance}.
% TODO: Vorteil des Impedanzreglers gegen diese Konzepte.
% Impedance Control can be seen as a generalized low-level concept to ensure compliance and can be combined with other high-level concepts.

%% Sicheres Verhalten auch durch Reflexe (zusätzlich zur Regelung)
% Voraussetzungen für Reflexverhalten
Activating reflex reactions as a response to potentially unwanted environmental contacts is another main pillar in safe and sensitive robot interaction.
This requires the ability to discriminate internal from external torques based on accurate dynamics models together with proprioceptive position and torque measurements.
% Möglichkeit zur Umsetzung: Beobachter
Disturbance observers are a common technique in robotics to handle either modeling inaccuracies \cite{Oh1999,NikoobinHag2008} or to recognize unexpected events such as collisions.
% Beispiele zur Anwendung
\cite{LucaMat2005} used a momentum-based disturbance observer for \textbf{collision detection}, \textbf{isolation}, and \textbf{estimation}, including validation with a 2-Degree-of-Freedom (DoF) simulation.
These results were extended to the flexible manipulator case and experimentally validated with the DLR lightweight robot arm III~\cite{DeLucaAlbHadHir2006}, using the concepts of \emph{total link energy} and \emph{generalized momentum}.
An analysis of different terms in the error dynamics and an approach for velocity-based variable collision thresholds were presented in \cite{SotoudehnejadKer2014}.
An estimation of the external end-effector wrenches based on observed disturbance torques was used in \cite{OttHenLee2013} to enhance a model predictive balancing controller on the humanoid robot TORO.

% Erweiterungen zu Kollisionserkennung: Einordnung der Kollisionen
In \cite{GolzOseHad2015} the subsequent \textbf{collision classification} problem was approached by applying state-of-the-art machine learning techniques to learning the collision torque profiles of different collision types, including features based on collision frequencies, amplitudes or other physically motivated aspects.
A summary of collision handling can be found in \cite{Haddadin2014}.
The focus of these works was on electromechanically actuated robots equipped with link-side joint torque sensing.

% Anwendungsbeispiel Hydraulische Roboter
A collision detection for the 6-DoF hydraulic robot arm \emph{Maestro} with low-pass filtered model error was evaluated in \cite{BidardLibArhMea2005}, focussing on hydraulic friction effects.
In \cite{jung2012collision} the authors applied a disturbance observer for collision detection that contains a bandpass filter, making use of specifically identified collision frequencies, to a 3-DoF hydraulic robot arm with joint torque sensors. 

% Unterschiede elektromechanisch, hydraulisch, Herausforderungen
In contrast to their electromechanical counterpart, and due to the fact that no high gearing with according friction losses is required, commonly used hydraulic actuators do not have link-side torque measurements. 
In fact, actuator forces can be estimated via the chamber pressures. 
However, link-side friction, e.g. induced by the chamber seals, can be rather high indeed.
This in turn makes direct high-fidelity joint torque control, and thus also any other soft-robotics control concept, difficult to realize.

%% Unser Beitrag
% Identifikation, Überleitung von Herausforderungen bei Hydraulik zu unserem Ansatz (getrennte Identifikation)
In the present work, we approach this problem by identifying friction and the required dynamic parameters for our model-based control concept with a two-step method.
In the first step, a friction model including Coulomb and viscous friction is identified.
In the second step, the results of the first step are used to reduce the number of parameters for the identification of the dynamics model.

% Störgrößenbeobachter zur Modellfehlerkompensation: Vorteile, Nachteile, Ausgleich der Nachteile
In order to compensate for model errors when using impedance control, a generalized momentum-based disturbance observer is used \cite{SpVdTh14}. 
As a drawback compliance is lost, since external forces cannot be systematically discriminated from the disturbance torques and are thus compensated by the observer as well.
To overcome this problem for end-effector contacts
we use a wrist force/torque sensor to calculate the external joint torques and exclude them from the disturbance torque calculation.
% Störgrößenbeobachter zur Schätzung externer Kräfte: Erweiterungen
Finally, an identified friction model is part of the control scheme to further reduce the observed disturbances.

% Zusammenfassung unseres Beitrages
The contributions of this paper are
\begin{enumerate}
\item the extension of momentum-based collision handling by considering effects such as friction and wrist force/torque measurements, % methodische Erweiterung der Kollisionserkennung
\item the experimental validation of disturbance observer-enhanced collision detection and reaction schemes to the hydraulic part of the 7-DoF Atlas robot arms,
\item simulative results of a disturbance observer compensation concept based on wrist force/torque measurement increasing the impedance controller precision and keeping the end-effector compliance simultaneously, and
\item a two-step friction and extended rigid body dynamics identification method applied to a serial chain robot with both hydraulic and electromechanic actuators.
\end{enumerate}
% Aufbau des Artikels
The paper is organized as follows.
Section~\ref{sec:advance} adapts our identification and control concepts from \cite{SpVdTh14} and compares the results to the previous approach.
Section~\ref{sec:CollDet} shows experimental results for collision detection and reaction with the Atlas robot.
Furthermore, the concept of regaining compliance when using disturbance compensation from \cite{Oh1999} for our controller implementation is explained.
Section~\ref{sec:conclusion} concludes the paper.

