% !TEX root = atlas_iros_16.tex
%%%%%%%%%%%%%%%%%%%%%%%%%%%%%%%%%%%%%%%%%%%%%%%%%%%%%%%%%%%%%%%%%%%%%%%%%%%%%%%%
\section{\large Conclusion}
\label{sec:conclusion}

%\textbf{Schreibt die Conclusion bitte in der Art: In diesem Paper haben wir folgende Methoden erarbeitet, die es erlaubt haben folgendes Verhalten zu ermoeglichen. Dies erlaubt... Bisherige Limitierungen sind und koennen folgendermassen ind er Zukunft untersucht und geloest werden.}
In this paper we designed and implemented a two-step identification method and an adapted generalized momentum observer for articulated robots.
We applied both methods to the arms of the humanoid robot Atlas. 
The two-step identification significantly improves the identification results in the presence of high friction in the examined scenario compared to our previous work. 
The adapted observer enables precise and compliant manipulation.
In addition, friction models can be incorporated to further improve the observer performance. 
The proposed solution is able to detect collisions along the entire robot structure. 
Overall, this provides key features to safely operate in unknown environments and is a first step towards the safe cooperation of (partly) hydraulically driven humanoids with humans.

%One Limitation of the current approach is the still rather basic friction model.
%Also, more sophisticated reaction strategies beyond switching to gravity compensation are to be developed.

%%%%%%%%%%%%%%%%%%%%%%%%%%%%%%%%%%%%%%%%%%%%%%%%%%%%%%%%%%%%%%%%%%%%%%%%%%%%%%%%
%\section*{APPENDIX}

%Appendixes should appear before the acknowledgment.


%%%%%%%%%%%%%%%%%%%%%%%%%%%%%%%%%%%%%%%%%%%%%%%%%%%%%%%%%%%%%%%%%%%%%%%%%%%%%%%%

% BIBLIOGRAPHY
\renewcommand{\baselinestretch}{\BibBaselineStretch}
\bibliographystyle{ieeetr}
\bibliography{iros_references} 
